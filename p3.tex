\title{TartanTrade - A Distributed Marketplace}
\author{
        Mat Gray \\
        mhgray\\
            \and
        Miko Bautista\\
        mbautist\\
}
\date{April 15, 2014}

\documentclass[12pt]{article}

\begin{document}
\maketitle

\section{Description}
\hspace{\fill}

TartanTrades is a system for batteling the corrupt block system at CMU\@.  Currently, undergraduates are given "blocks" to be used to purchase meals.
These blocks expire every other week, and many go to waste since most students are unable to consume all of the blocks within the tight time limit.\\

We will solve this problem by creating a distributed marketplace that will allow for students to buy/sell their blocks.  Someone who is stuck on the meal
plan can put their blocks up for sale with a certain asking price.  Other students can mark themselves as "buyers" with a maximum amount that they are willing
to pay for the block as well as a maximum distance that they are willing to walk.  If a student wishes to purchase a block, all of the sellers are notified with
the option of claiming the sale.

There will be a rating system for buyers/sellers to ensure that people don't take advantage of others and not pay or not have the block.\\


\section{Implementation Structure}


\section{Distributed Features \& Algorithms}

TartanTrades will use Lamport timestamps in order to determine the order of events that happen.  For example, it will determine who reserved a block first.

TartanTrades will also make use of Paxos for handling server node failures and also for ensuring exactly one buyer is connected to one seller.

\section{Testing}


\section{Tiers}


\section{Schedule}

\end{document}
